\documentclass[a4paper,11pt] {article}

\setlength{\hoffset}{-18pt}         
\setlength{\oddsidemargin}{9pt} % Marge gauche sur pages impaires
\setlength{\evensidemargin}{9pt} % Marge gauche sur pages paires
\setlength{\marginparwidth}{54pt} % Largeur de note dans la marge
\setlength{\textwidth}{481pt} % Largeur de la zone de texte (17cm)
\setlength{\voffset}{-18pt} % Bon pour DOS
\setlength{\marginparsep}{7pt} % Séparation de la marge
\setlength{\topmargin}{0pt} % Pas de marge en haut
\setlength{\headheight}{13pt} % Haut de page
\setlength{\headsep}{10pt} % Entre le haut de page et le texte
\setlength{\footskip}{27pt} % Bas de page + séparation
\setlength{\textheight}{708pt} % Hauteur de la zone de texte (25cm)

\usepackage[utf8]{inputenc}
\usepackage[francais]{babel}
\usepackage[T1]{fontenc}
\usepackage{lmodern} % load a font with all the characters
\usepackage{mathtools}
\usepackage{dsfont}
\usepackage{amsmath}
\usepackage{graphicx}
\usepackage[export]{adjustbox}
\usepackage{listings}
\usepackage{amsthm}
\author{HASSAN Lucas, JACQUOT Paulin}
\date{décembre 2015}
\usepackage{amsfonts}
\usepackage{stmaryrd}

\begin{document}


\title{Introduction à l'API C++ de Cplex}

\maketitle{}
\section{Exerci 1}
\subsection{Modélisation naïve du problème}

Le problème se modélise naturellement de la façon suivant : 

\begin{equation}
\begin{array}{l}
\max \sum\limits_{(i,j)\in M} x_i{ij}\\
H^p(M)+H^a(M) \geq 2
\end{array}
\end{equation}

Où $H^p(M) = \frac{\sum\limits_{(i,j)\in M} H^p_{ij}C^p_{ij}x_{ij}}{\sum\limits_{(i,j)\in M} C^p_{ij}x_{ij}}$ et $H^a(M) = \frac{\sum\limits_{(i,j)\in M} H^p_{ij}C^a_{ij}x_{ij}}{\sum\limits_{(i,j)\in M} C^a_{ij}x_{ij}}$

On remarque que la contrainte définie est fractionnaire. Ainsi on choisit la linéariser de la manière suivante : 

Cette contrainte s'écrit aussi : 

\begin{equation}
\begin{array}{l}
\sum\limits_{(i,j)\in M} H^p_{ij}C^p_{ij}x_{ij}(\sum\limits_{(i,j)\in M} C^p_{ij}x_{ij}) + \sum\limits_{(i,j)\in M} H^a_{ij}C^a_{ij}x_{ij}(\sum\limits_{(i,j)\in M} C^a_{ij}x_{ij}) \geq 2(\sum\limits_{(k,l)\in M} C^a_{kl}x_{kl})(\sum\limits_{(k,l)\in M} C^p_{kl}x_{kl})\\
\end{array}
\end{equation}

Définissons les variables réelles $p$ et $a$ telle que  : 
\begin{equation}
\begin{array}{l}
p = (\sum\limits_{(i,j)\in M} C^p_{ij}x_{ij})\\
a = (\sum\limits_{(i,j)\in M} C^a_{ij}x_{ij})
\end{array}
\end{equation}

Les contraintes s'écrivent alors : 
\begin{equation}
\begin{array}{l}
\sum\limits_{(i,j)\in M} H^p_{ij}C^p_{ij}px_{ij} + \sum\limits_{(i,j)\in M} H^a_{ij}C^a_{ij}ax_{ij} \geq 2(\sum\limits_{(i,j)\in M} C^a_{ij}px_{ij})\\
p = (\sum\limits_{(i,j)\in M} C^p_{ij}x_{ij})
a = (\sum\limits_{(i,j)\in M} C^a_{ij}x_{ij})
\end{array}
\end{equation}

On définit ensuite $\forall\quad (i,j)\in M a_{ij}=ax_{ij}$ et $p_{ij}=px_{ij}$

On sait que l'égualité $a_{ij}=ax_{ij}$ s'écris encore : 
\begin{equation}
\begin{array}{l}
a_{ij}\leq a_{max}x_{ij}\\
a_{ij}\leq a\\
a_{ij}\geq (x_{ij}-1)a_{max}+a\\
a_{ij} \geq 0
\end{array}
\end{equation}

où $a_{max}$ est une borne supérieur de $a$. Une borne supérieur triviale de $a$ est ici $\sum\limits_{(i,j)\in M} C^a_{ij}$.

Ainsi, on définit le problème linéarisé : 

\begin{equation}
\begin{array}{l}
\max \sum\limits_{(i,j)\in M} x_{ij}\\
\sum\limits_{(i,j)\in M} H^p_{ij}C^p_{ij}p_{ij} + \sum\limits_{(i,j)\in M} H^a_{ij}C^a_{ij}a_{ij} \geq 2(\sum\limits_{(i,j)\in M} C^a_{ij}p_{ij})\\
a_{ij}\leq a_{max}x_{ij}\\
a_{ij}\leq a\\
a_{ij}\geq (x_{ij}-1)a_{max}+a\\
a_{ij} \geq 0\\
p_{ij}\leq p_{max}x_{ij}\\
p_{ij}\leq p\\
p_{ij}\geq (x_{ij}-1)p_{max}+p\\
p_{ij} \geq 0\\
p = (\sum\limits_{(i,j)\in M} C^p_{ij}x_{ij})\\
a = (\sum\limits_{(i,j)\in M} C^a_{ij}x_{ij})
\end{array}
\end{equation}
\end{document}